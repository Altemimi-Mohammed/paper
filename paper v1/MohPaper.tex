
%% bare_conf.tex
%% V1.3
%% 2007/01/11
%% by Michael Shell
%% See:
%% http://www.michaelshell.org/
%% for current contact information.
%%
%% This is a skeleton file demonstrating the use of IEEEtran.cls
%% (requires IEEEtran.cls version 1.7 or later) with an IEEE conference paper.
%%
%% Support sites:
%% http://www.michaelshell.org/tex/ieeetran/
%% http://www.ctan.org/tex-archive/macros/latex/contrib/IEEEtran/
%% and
%% http://www.ieee.org/

%%*************************************************************************
%% Legal Notice:
%% This code is offered as-is without any warranty either expressed or
%% implied; without even the implied warranty of MERCHANTABILITY or
%% FITNESS FOR A PARTICULAR PURPOSE! 
%% User assumes all risk.
%% In no event shall IEEE or any contributor to this code be liable for
%% any damages or losses, including, but not limited to, incidental,
%% consequential, or any other damages, resulting from the use or misuse
%% of any information contained here.
%%
%% All comments are the opinions of their respective authors and are not
%% necessarily endorsed by the IEEE.
%%
%% This work is distributed under the LaTeX Project Public License (LPPL)
%% ( http://www.latex-project.org/ ) version 1.3, and may be freely used,
%% distributed and modified. A copy of the LPPL, version 1.3, is included
%% in the base LaTeX documentation of all distributions of LaTeX released
%% 2003/12/01 or later.
%% Retain all contribution notices and credits.
%% ** Modified files should be clearly indicated as such, including  **
%% ** renaming them and changing author support contact information. **
%%
%% File list of work: IEEEtran.cls, IEEEtran_HOWTO.pdf, bare_adv.tex,
%%                    bare_conf.tex, bare_jrnl.tex, bare_jrnl_compsoc.tex
%%*************************************************************************

% *** Authors should verify (and, if needed, correct) their LaTeX system  ***
% *** with the testflow diagnostic prior to trusting their LaTeX platform ***
% *** with production work. IEEE's font choices can trigger bugs that do  ***
% *** not appear when using other class files.                            ***
% The testflow support page is at:
% http://www.michaelshell.org/tex/testflow/



% Note that the a4paper option is mainly intended so that authors in
% countries using A4 can easily print to A4 and see how their papers will
% look in print - the typesetting of the document will not typically be
% affected with changes in paper size (but the bottom and side margins will).
% Use the testflow package mentioned above to verify correct handling of
% both paper sizes by the user's LaTeX system.
%
% Also note that the "draftcls" or "draftclsnofoot", not "draft", option
% should be used if it is desired that the figures are to be displayed in
% draft mode.
%
\documentclass[conference, compsoc]{IEEEtran}
% Add the compsoc option for Computer Society conferences.
%
% If IEEEtran.cls has not been installed into the LaTeX system files,
% manually specify the path to it like:
% \documentclass[conference]{../sty/IEEEtran}





% Some very useful LaTeX packages include:
% (uncomment the ones you want to load)


% *** MISC UTILITY PACKAGES ***
%
%\usepackage{ifpdf}
% Heiko Oberdiek's ifpdf.sty is very useful if you need conditional
% compilation based on whether the output is pdf or dvi.
% usage:
% \ifpdf
%   % pdf code
% \else
%   % dvi code
% \fi
% The latest version of ifpdf.sty can be obtained from:
% http://www.ctan.org/tex-archive/macros/latex/contrib/oberdiek/
% Also, note that IEEEtran.cls V1.7 and later provides a builtin
% \ifCLASSINFOpdf conditional that works the same way.
% When switching from latex to pdflatex and vice-versa, the compiler may
% have to be run twice to clear warning/error messages.






% *** CITATION PACKAGES ***
%
%\usepackage{cite}
% cite.sty was written by Donald Arseneau
% V1.6 and later of IEEEtran pre-defines the format of the cite.sty package
% \cite{} output to follow that of IEEE. Loading the cite package will
% result in citation numbers being automatically sorted and properly
% "compressed/ranged". e.g., [1], [9], [2], [7], [5], [6] without using
% cite.sty will become [1], [2], [5]--[7], [9] using cite.sty. cite.sty's
% \cite will automatically add leading space, if needed. Use cite.sty's
% noadjust option (cite.sty V3.8 and later) if you want to turn this off.
% cite.sty is already installed on most LaTeX systems. Be sure and use
% version 4.0 (2003-05-27) and later if using hyperref.sty. cite.sty does
% not currently provide for hyperlinked citations.
% The latest version can be obtained at:
% http://www.ctan.org/tex-archive/macros/latex/contrib/cite/
% The documentation is contained in the cite.sty file itself.






% *** GRAPHICS RELATED PACKAGES ***
%
\ifCLASSINFOpdf
  % \usepackage[pdftex]{graphicx}
  % declare the path(s) where your graphic files are
  % \graphicspath{{../pdf/}{../jpeg/}}
  % and their extensions so you won't have to specify these with
  % every instance of \includegraphics
  % \DeclareGraphicsExtensions{.pdf,.jpeg,.png}
\else
  % or other class option (dvipsone, dvipdf, if not using dvips). graphicx
  % will default to the driver specified in the system graphics.cfg if no
  % driver is specified.
  % \usepackage[dvips]{graphicx}
  % declare the path(s) where your graphic files are
  % \graphicspath{{../eps/}}
  % and their extensions so you won't have to specify these with
  % every instance of \includegraphics
  % \DeclareGraphicsExtensions{.eps}
\fi
% graphicx was written by David Carlisle and Sebastian Rahtz. It is
% required if you want graphics, photos, etc. graphicx.sty is already
% installed on most LaTeX systems. The latest version and documentation can
% be obtained at: 
% http://www.ctan.org/tex-archive/macros/latex/required/graphics/
% Another good source of documentation is "Using Imported Graphics in
% LaTeX2e" by Keith Reckdahl which can be found as epslatex.ps or
% epslatex.pdf at: http://www.ctan.org/tex-archive/info/
%
% latex, and pdflatex in dvi mode, support graphics in encapsulated
% postscript (.eps) format. pdflatex in pdf mode supports graphics
% in .pdf, .jpeg, .png and .mps (metapost) formats. Users should ensure
% that all non-photo figures use a vector format (.eps, .pdf, .mps) and
% not a bitmapped formats (.jpeg, .png). IEEE frowns on bitmapped formats
% which can result in "jaggedy"/blurry rendering of lines and letters as
% well as large increases in file sizes.
%
% You can find documentation about the pdfTeX application at:
% http://www.tug.org/applications/pdftex





% *** MATH PACKAGES ***
%
%\usepackage[cmex10]{amsmath}
% A popular package from the American Mathematical Society that provides
% many useful and powerful commands for dealing with mathematics. If using
% it, be sure to load this package with the cmex10 option to ensure that
% only type 1 fonts will utilized at all point sizes. Without this option,
% it is possible that some math symbols, particularly those within
% footnotes, will be rendered in bitmap form which will result in a
% document that can not be IEEE Xplore compliant!
%
% Also, note that the amsmath package sets \interdisplaylinepenalty to 10000
% thus preventing page breaks from occurring within multiline equations. Use:
%\interdisplaylinepenalty=2500
% after loading amsmath to restore such page breaks as IEEEtran.cls normally
% does. amsmath.sty is already installed on most LaTeX systems. The latest
% version and documentation can be obtained at:
% http://www.ctan.org/tex-archive/macros/latex/required/amslatex/math/





% *** SPECIALIZED LIST PACKAGES ***
%
%\usepackage{algorithmic}
% algorithmic.sty was written by Peter Williams and Rogerio Brito.
% This package provides an algorithmic environment fo describing algorithms.
% You can use the algorithmic environment in-text or within a figure
% environment to provide for a floating algorithm. Do NOT use the algorithm
% floating environment provided by algorithm.sty (by the same authors) or
% algorithm2e.sty (by Christophe Fiorio) as IEEE does not use dedicated
% algorithm float types and packages that provide these will not provide
% correct IEEE style captions. The latest version and documentation of
% algorithmic.sty can be obtained at:
% http://www.ctan.org/tex-archive/macros/latex/contrib/algorithms/
% There is also a support site at:
% http://algorithms.berlios.de/index.html
% Also of interest may be the (relatively newer and more customizable)
% algorithmicx.sty package by Szasz Janos:
% http://www.ctan.org/tex-archive/macros/latex/contrib/algorithmicx/




% *** ALIGNMENT PACKAGES ***
%
%\usepackage{array}
% Frank Mittelbach's and David Carlisle's array.sty patches and improves
% the standard LaTeX2e array and tabular environments to provide better
% appearance and additional user controls. As the default LaTeX2e table
% generation code is lacking to the point of almost being broken with
% respect to the quality of the end results, all users are strongly
% advised to use an enhanced (at the very least that provided by array.sty)
% set of table tools. array.sty is already installed on most systems. The
% latest version and documentation can be obtained at:
% http://www.ctan.org/tex-archive/macros/latex/required/tools/


%\usepackage{mdwmath}
%\usepackage{mdwtab}
% Also highly recommended is Mark Wooding's extremely powerful MDW tools,
% especially mdwmath.sty and mdwtab.sty which are used to format equations
% and tables, respectively. The MDWtools set is already installed on most
% LaTeX systems. The lastest version and documentation is available at:
% http://www.ctan.org/tex-archive/macros/latex/contrib/mdwtools/


% IEEEtran contains the IEEEeqnarray family of commands that can be used to
% generate multiline equations as well as matrices, tables, etc., of high
% quality.


%\usepackage{eqparbox}
% Also of notable interest is Scott Pakin's eqparbox package for creating
% (automatically sized) equal width boxes - aka "natural width parboxes".
% Available at:
% http://www.ctan.org/tex-archive/macros/latex/contrib/eqparbox/





% *** SUBFIGURE PACKAGES ***
%\usepackage[tight,footnotesize]{subfigure}
% subfigure.sty was written by Steven Douglas Cochran. This package makes it
% easy to put subfigures in your figures. e.g., "Figure 1a and 1b". For IEEE
% work, it is a good idea to load it with the tight package option to reduce
% the amount of white space around the subfigures. subfigure.sty is already
% installed on most LaTeX systems. The latest version and documentation can
% be obtained at:
% http://www.ctan.org/tex-archive/obsolete/macros/latex/contrib/subfigure/
% subfigure.sty has been superceeded by subfig.sty.



%\usepackage[caption=false]{caption}
%\usepackage[font=footnotesize]{subfig}
% subfig.sty, also written by Steven Douglas Cochran, is the modern
% replacement for subfigure.sty. However, subfig.sty requires and
% automatically loads Axel Sommerfeldt's caption.sty which will override
% IEEEtran.cls handling of captions and this will result in nonIEEE style
% figure/table captions. To prevent this problem, be sure and preload
% caption.sty with its "caption=false" package option. This is will preserve
% IEEEtran.cls handing of captions. Version 1.3 (2005/06/28) and later 
% (recommended due to many improvements over 1.2) of subfig.sty supports
% the caption=false option directly:
%\usepackage[caption=false,font=footnotesize]{subfig}
%
% The latest version and documentation can be obtained at:
% http://www.ctan.org/tex-archive/macros/latex/contrib/subfig/
% The latest version and documentation of caption.sty can be obtained at:
% http://www.ctan.org/tex-archive/macros/latex/contrib/caption/




% *** FLOAT PACKAGES ***
%
%\usepackage{fixltx2e}
% fixltx2e, the successor to the earlier fix2col.sty, was written by
% Frank Mittelbach and David Carlisle. This package corrects a few problems
% in the LaTeX2e kernel, the most notable of which is that in current
% LaTeX2e releases, the ordering of single and double column floats is not
% guaranteed to be preserved. Thus, an unpatched LaTeX2e can allow a
% single column figure to be placed prior to an earlier double column
% figure. The latest version and documentation can be found at:
% http://www.ctan.org/tex-archive/macros/latex/base/



%\usepackage{stfloats}
% stfloats.sty was written by Sigitas Tolusis. This package gives LaTeX2e
% the ability to do double column floats at the bottom of the page as well
% as the top. (e.g., "\begin{figure*}[!b]" is not normally possible in
% LaTeX2e). It also provides a command:
%\fnbelowfloat
% to enable the placement of footnotes below bottom floats (the standard
% LaTeX2e kernel puts them above bottom floats). This is an invasive package
% which rewrites many portions of the LaTeX2e float routines. It may not work
% with other packages that modify the LaTeX2e float routines. The latest
% version and documentation can be obtained at:
% http://www.ctan.org/tex-archive/macros/latex/contrib/sttools/
% Documentation is contained in the stfloats.sty comments as well as in the
% presfull.pdf file. Do not use the stfloats baselinefloat ability as IEEE
% does not allow \baselineskip to stretch. Authors submitting work to the
% IEEE should note that IEEE rarely uses double column equations and
% that authors should try to avoid such use. Do not be tempted to use the
% cuted.sty or midfloat.sty packages (also by Sigitas Tolusis) as IEEE does
% not format its papers in such ways.





% *** PDF, URL AND HYPERLINK PACKAGES ***
%
%\usepackage{url}
% url.sty was written by Donald Arseneau. It provides better support for
% handling and breaking URLs. url.sty is already installed on most LaTeX
% systems. The latest version can be obtained at:
% http://www.ctan.org/tex-archive/macros/latex/contrib/misc/
% Read the url.sty source comments for usage information. Basically,
% \url{my_url_here}.





% *** Do not adjust lengths that control margins, column widths, etc. ***
% *** Do not use packages that alter fonts (such as pslatex).         ***
% There should be no need to do such things with IEEEtran.cls V1.6 and later.
% (Unless specifically asked to do so by the journal or conference you plan
% to submit to, of course. )


% correct bad hyphenation here
\hyphenation{op-tical net-works semi-conduc-tor}


\begin{document}
%
% paper title
% can use linebreaks \\ within to get better formatting as desired
\title{Assignment 2 - Research Paper }


% author names and affiliations
% use a multiple column layout for up to two different
% affiliations

\author{\IEEEauthorblockN{Mohammed Altemimi (P47901)}
\IEEEauthorblockA{dept. FTSM\\
Mohammed Alaa Hussein Altemimi\\
UKM University , Malaysia \\
goodprogrammer2005@yahoo.com}
\and
\IEEEauthorblockN{}
\IEEEauthorblockA{\\ }}

% conference papers do not typically use \thanks and this command
% is locked out in conference mode. If really needed, such as for
% the acknowledgment of grants, issue a \IEEEoverridecommandlockouts
% after \documentclass

% for over three affiliations, or if they all won't fit within the width
% of the page, use this alternative format:
% 
%\author{\IEEEauthorblockN{Michael Shell\IEEEauthorrefmark{1},
%Homer Simpson\IEEEauthorrefmark{2},
%James Kirk\IEEEauthorrefmark{3}, 
%Montgomery Scott\IEEEauthorrefmark{3} and
%Eldon Tyrell\IEEEauthorrefmark{4}}
%\IEEEauthorblockA{\IEEEauthorrefmark{1}School of Electrical and Computer Engineering\\
%Georgia Institute of Technology,
%Atlanta, Georgia 30332--0250\\ Email: see http://www.michaelshell.org/contact.html}
%\IEEEauthorblockA{\IEEEauthorrefmark{2}Twentieth Century Fox, Springfield, USA\\
%Email: homer@thesimpsons.com}
%\IEEEauthorblockA{\IEEEauthorrefmark{3}Starfleet Academy, San Francisco, California 96678-2391\\
%Telephone: (800) 555--1212, Fax: (888) 555--1212}
%\IEEEauthorblockA{\IEEEauthorrefmark{4}Tyrell Inc., 123 Replicant Street, Los Angeles, California 90210--4321}}




% use for special paper notices
%\IEEEspecialpapernotice{(Invited Paper)}




% make the title area
\maketitle


\begin{abstract}
%\boldmath
This paper describe on the study of cloud computing in current IT environment. It is just a preliminary study on the overview of cloud computing and what it was. In this paper, the history, benefit and security aspect in cloud computing will be explained. In addition, it will also touch about the different between cloud computing and others computer system. Besides that, it will try to elaborate on the issue of how cloud computing will change the current business and what will be happen in the future. 

\end{abstract}
% IEEEtran.cls defaults to using nonbold math in the Abstract.
% This preserves the distinction between vectors and scalars. However,
% if the conference you are submitting to favors bold math in the abstract,
% then you can use LaTeX's standard command \boldmath at the very start
% of the abstract to achieve this. Many IEEE journals/conferences frown on
% math in the abstract anyway.

% no keywords




% For peer review papers, you can put extra information on the cover
% page as needed:
% \ifCLASSOPTIONpeerreview
% \begin{center} \bfseries EDICS Category: 3-BBND \end{center}
% \fi
%
% For peerreview papers, this IEEEtran command inserts a page break and
% creates the second title. It will be ignored for other modes.
\IEEEpeerreviewmaketitle



\section{Introduction}
% no \IEEEPARstart
Cloud computing is a new technology where application developers and IT related service providers distributing their product over the Internet. The distribution is either free or chargeable. The user can subscribe either pay-per-use basis or monthly basis payment. The benefit of cloud computing is that the user do not have to install the application that their want to use into their computer.

The application distributed via cloud computing can be accessed via internet at anytime thus eliminate the worries of software maintenance. The software was maintained in the server by the service provider without the concern of the user. Large and powerful servers were used to ensure the continuous streams of data are channel to the user without fail. With cloud computing, IT related business and software can be billed like public utilities such as. electricity and water. 

From Wikipedia,the free encyclopedia is Internet-based ("cloud") development and use of computer technology ("computing"). The cloud is a metaphor for the Internet (based on how it is depicted in computer network diagrams) and is an abstraction for the complex infrastructure it conceals.[1] It is a style of computing in which IT-related capabilities are provided "as a service",[2] allowing users to access technology-enabled services from the Internet ("in the cloud")[3] without knowledge of, expertise with, or control over the technology infrastructure that supports them.[4] According to a 2008 paper published by IEEE Internet Computing "Cloud Computing is a paradigm in which information is permanently stored in servers on the Internet and cached temporarily on clients that include desktops, entertainment centers, table computers, notebooks, wall computers, handhelds, sensors, monitors, etc."[5]

Cloud computing is a general concept that incorporates software as a service (SaaS), Web 2.0 and other recent, well-known technology trends, in which the common theme is reliance on the Internet for satisfying the computing needs of the users. For example, Google Apps provides common business applications online that are accessed from a web browser, while the software and data are stored on the servers.
\section{BACKGROUND}

Cloud computing is an architecture that gives a whole new meaning to software as a service and it give the Internet a whole new meaning. The whole architecture rely on Internet to server their user and to satisfy the computing needs of the user. Cloud computing allow application service provider to provide application online that can be accessed through web browser, while the software and data are stored on the server.

In this new system, there will be significantly change on the workload at the user side. The user as the local computer can reduce the need of heavy computer load either software or hardware side. This requirement had been transferred to the computers in the networks and handle by themselves. Therefore, hardware and software demand on the local computer will be decrease and the only requirement is the ability of computer to run the cloud computing system's interface software.  It would be not a problem to run this software which is simple as common web browser. 

We may not aware that we have similar experience on using the cloud computing concept. The web base email service like Hotmail, Yahoo!, and Gmail are some of example of cloud computing concept. Instead of running an email program in the computer, the above email can log in through the web site as long as there is internet connection. All the data and emails are not saved in the personal computer but it was located in the cloud computer services.


In cloud computing, one of the primary benefits is the speed, where by people can get the services and bypass traditional IT departments. Figure 1 show the Latest Evolution Of Hosting in cloud computing. Cloud computing differs from existing hosting services. The hosting services are based on consumption and the technology of the infrastructure and it was optimized to serve several customers. At the same time the providers use virtualization extensively and grid computing software.

Forrester research have identified several companies as "cloud providers," including Amazon.com, Akamai Technologies, Joyent,and Rackspac's Mosso software. On the other hand Microsoft and Google are also rumored to be developing a computing services on usage basis, such as hosted server processing and storage. As these providers are optimized for large-scale hosts, they could eventually serve corporate customers.

Beside that, cost factor is one of the reason for this cloud computing development. In one organisation with quite numbers of employees, the IT administrator had to ensure all the employees have the right software and hardware they need for their jobs. Ideal solution is to purchase computers for everyone with the right software. One thing to remember, each software come with licence requirement. However this solution required huge money and the more employees means the more money they need to invest. Therefore, cloud computing is the correct solution which the only need is to install one application. In this application, each user are allowed to log in to a web-based service which hosts all the programs the user need for his or her jobs. 


\section{HISTORY}
The idea of cloud computing was started way back to 1960 by John McCarty who opined that "Computing may someday be organized as a public utility". Starting by the year 2000, some big Internet base company such as Amazon.com has shown their interest in cloud computing. At that time most of the interested companies only focus on software as a service. 

In 2007, the activities involving cloud computing have greatly increased. Company like Google and IBM and also some number of universities have invested a great amount of money into the development and research of cloud computing.

\vspace{.5cm}	

\section{CLOUD COMPUTING VS GRID COMPUTING}
Cloud computing is not like Grid Computing. Grid Computing is where a group of computers is linked together in a network to performed one very large task. Cloud computing on the other hand, can perform multiple task. It is like a server hosting for multiple application. 

Grid computing has been use in current market where users make a few request which allocate large requirement each. For example, an organization may have 1000  node cluster and group it to few allocation, let say 200.  Unfortunately, only a few allocation can be serve at one time where by the others have to wait and may need to reschedule for other time when the resources are released. This phenomena will results a batch job scheduling algorithms of parallel computations.[1]

Cloud computing is concentrate to small allocation requests. For example Amazon EC2 accounts are limited to 20 servers each and a lot of users allocate up to 20 servers from many thousands of servers someone else release resources. This situation is completely different resource allocation paradigm, usage pattern and results in completely different method of using computer resources. 

Anyway, it not that easy to create a cloud even though it just consist a cloud management software and support by a few computers. On the other hand, it is a challenging to uphold the premise of real time resource availability. The cloud provider need to provide resources and if they fail, the whole system will collapse and the users will start hoarding servers, resulting a peak usage than normal demand.

Below table shows the summary of comparison between cloud computing and grid computing by using a model from each category.

\section{BENEFITS}

There are many benefits in cloud computing systems. All user either individual or organisation can enjoy many advantages when apply to this new systems. Simple and minimum IT management with low cost requirement makes cloud computing are practically to be use for all kind of user. That is why this new practice is now in rapid development and being popular in IT environment.  

\textbf{
\begin{itemize}
\item Client or users can access their application and data from anywhere at any time. The only needs to be linked to cloud computing system is a computer with internet connection.
\end{itemize}
}

\textbf{
\begin{itemize}
\item The needs of advance hardware and software are no longer required and could be reduce the hardware and software cost. It means, the user are no longer need to buy the fastest computer with huge memory because the cloud computer system will take care all this requirement. Therefore all the user needs could be a cheaper personal computer with basic and common requirement which can easily get it from the market. A simple netbook with small in size and weight is one example that can fulfill this requirement  with very low price ( cost about RM 1200 ~ 1800 ). All this thing is the result of a system that the user doesn't need a large memory and disk space because all the user data are stored on a remote computer.
\end{itemize}
}

\textbf{
\begin{itemize}
\item In normal business situation, an organization have to make sure they have the right software for their business operations. It might  be a common software that use by all their employee. In the cloud computing systems, they can provide this organization an access to their computer application for all the employee. Therefore the need of a set of software with license for all the employee are no longer required. Instead, the company are only required to pay the access fee to the cloud computing company.
\end{itemize}
}

\textbf{
\begin{itemize}
\item Some organization need to provide some space for their server  and digital storage. However cloud computing can offer to this organization the option of storing data on someone else hardware. This phenomena will remove the need for physical space on the front end.
\end{itemize}
}


\textbf{
\begin{itemize}
\item The client could have advantages of the entire network's processing power if the cloud computing system's was supported by a grid computing system at the back end. There are many cases that a scientists or researchers works with very complex calculation and take so much time for individual computer to complete them. In this case, cloud computing system will tap into processing power for all available computers in the cloud and resulting reduce the calculation processing time.
\end{itemize}
}

\begin{center}
\section{HOW IT BEING USED}
\end{center}

In global IT environment cloud computing had been used widely besides we may not aware that sometimes we are part of it. Below is some example and how it being used by the user.

\textbf{
\begin{itemize}
\item SaaS
With the development of cloud computing come a new idea of distributing software through Internet. SaaS (Software as a service) is being introduced as a next step of software distribution. Using cloud computing architecture, a single application can be delivered through the browser to the user. On the user side there is no upfront investment in server of software licensing and on the provider side, they only have to maintain one application thus the maintenance cost will greatly reduce. Some examples of SaaS are Saleforce.com, Google Apps and Zoho Office.
\end{itemize}
}

\textbf{
\begin{itemize}
\item Utility computing
This form of cloud computing give a whole new meaning to IT related utility where software is charge just like a public utilities such as electricity or water. Company like Amazon.com, Sun, IBM and other who offer storage and virtual server and customers can access on demand. 
\end{itemize}
}


Figure 3 shows as the cloud computing hireiki and the providers. Second group (2) is lately call cloud computing. In first group (1),some of underlying work done that led cloud computing. At the top (3),are examples of each SaaS type.

\section{SECURITY IN CLOUD COMPUTING}

Here are seven of the specific security issues Gartner says customers should raise with vendors before selecting a cloud vendor.

\textbf{
\begin{itemize}
	\item Privileged user access.
\end{itemize}
}
Sensitive data processed outside the enterprise brings with it an inherent level of risk, because outsourced services bypass the "physical, logical and personnel controls" IT shops exert over in-house programs. Get as much information as you can about the people who manage your data. "Ask providers to supply specific information on the hiring and oversight of privileged administrators, and the controls over their access," Gartner says.

\textbf{
\begin{itemize}
	\item Regulatory compliance 
\end{itemize}
}
Customers are ultimately responsible for the security and integrity of their own data, even when it is held by a service provider. Traditional service providers are subjected to external audits and security certifications. Cloud computing providers who refuse to undergo this scrutiny are "signaling that customers can only use them for the most trivial functions," according to Gartner. 

\textbf{
\begin{itemize}
	\item Data location. 
\end{itemize}
}
When you use the cloud, you probably won't know exactly where your data is hosted. In fact, you might not even know what country it will be stored in. Ask providers if they will commit to storing and processing data in specific jurisdictions, and whether they will make a contractual commitment to obey local privacy requirements on behalf of their customers, Gartner advises.

\textbf{
\begin{itemize}
	\item Data segregation.
\end{itemize}
} 
Data in the cloud is typically in a shared environment alongside data from other customers. Encryption is effective but isn't a cure-all. "Find out what is done to segregate data at rest," Gartner advises. The cloud provider should provide evidence that encryption schemes were designed and tested by experienced specialists. "Encryption accidents can make data totally unusable, and even normal encryption can complicate availability," Gartner says.

\textbf{
\begin{itemize}
	\item Recovery. 
\end{itemize}
}
Even if you don't know where your data is, a cloud provider should tell you what will happen to your data and service in case of a disaster. "Any offering that does not replicate the data and application infrastructure across multiple sites is vulnerable to a total failure," Gartner says. Ask your provider if it has "the ability to do a complete restoration, and how long it will take."

\textbf{
\begin{itemize}
	\item Investigative support. 
\end{itemize}
}
Investigating inappropriate or illegal activity may be impossible in cloud computing, Gartner warns. "Cloud services are especially difficult to investigate, because logging and data for multiple customers may be co-located and may also be spread across an ever-changing set of hosts and data centers. If you cannot get a contractual commitment to support specific forms of investigation, along with evidence that the vendor has already successfully supported such activities, then your only safe assumption is that investigation and discovery requests will be impossible."

\textbf{
\begin{itemize}
	\item Long-term viability.
\end{itemize}
}
Ideally, your cloud computing provider will never go broke or get acquired and swallowed up by a larger company. But you must be sure your data will remain available even after such an event. "Ask potential providers how you would get your data back and if it would be in a format that you could import into a replacement application," Gartner says.


\section{CLOUD COMPUTING FOR MALAYSIAN GOVERMENT}
Cloud computing is still in its early stages, but the public sector is already beginning to see advantages. Despite its possible security and privacy risks, Cloud Computing  according to a magazine article due to be published later this Fall  has six main benefits that the public sector and government IT organizations are certain to want to take advantage of. In very brief summary form they are as follows:

\textbf{
\begin{itemize}
	\item Reduced Cost
\end{itemize}
}
Cloud technology is paid incrementally, saving organizations money. 

\textbf{
\begin{itemize}
	\item Increased Storage
\end{itemize}
}
Organizations can store more data than on private computer systems. 

\textbf{
\begin{itemize}
	\item Highly Automated 
\end{itemize}
}
No longer do IT personnel need to worry about keeping software up to date. 

\textbf{
\begin{itemize}
	\item Flexibility
\end{itemize}
}
Cloud computing offers much more flexibility than past computing methods. 

\textbf{
\begin{itemize}
	\item More Mobility 
\end{itemize}
}
Employees can access information wherever they are, rather than having to remain at their desks. 

\textbf{
\begin{itemize}
	\item Allows IT to Shift Focus
\end{itemize}
}
No longer having to worry about constant server updates and other computing issues, government organizations will be free to concentrate on innovation. 


% You must have at least 2 lines in the paragraph with the drop letter
% (should never be an issue)

\hfill 
 
\hfill September 1, 2009

% An example of a floating figure using the graphicx package.
% Note that \label must occur AFTER (or within) \caption.
% For figures, \caption should occur after the \includegraphics.
% Note that IEEEtran v1.7 and later has special internal code that
% is designed to preserve the operation of \label within \caption
% even when the captionsoff option is in effect. However, because
% of issues like this, it may be the safest practice to put all your
% \label just after \caption rather than within \caption{}.
%
% Reminder: the "draftcls" or "draftclsnofoot", not "draft", class
% option should be used if it is desired that the figures are to be
% displayed while in draft mode.
%
%\begin{figure}[!t]
%\centering
%\includegraphics[width=2.5in]{myfigure}
% where an .eps filename suffix will be assumed under latex, 
% and a .pdf suffix will be assumed for pdflatex; or what has been declared
% via \DeclareGraphicsExtensions.
%\caption{Simulation Results}
%\label{fig_sim}
%\end{figure}

% Note that IEEE typically puts floats only at the top, even when this
% results in a large percentage of a column being occupied by floats.


% An example of a double column floating figure using two subfigures.
% (The subfig.sty package must be loaded for this to work.)
% The subfigure \label commands are set within each subfloat command, the
% \label for the overall figure must come after \caption.
% \hfil must be used as a separator to get equal spacing.
% The subfigure.sty package works much the same way, except \subfigure is
% used instead of \subfloat.
%
%\begin{figure*}[!t]
%\centerline{\subfloat[Case I]\includegraphics[width=2.5in]{subfigcase1}%
%\label{fig_first_case}}
%\hfil
%\subfloat[Case II]{\includegraphics[width=2.5in]{subfigcase2}%
%\label{fig_second_case}}}
%\caption{Simulation results}
%\label{fig_sim}
%\end{figure*}
%
% Note that often IEEE papers with subfigures do not employ subfigure
% captions (using the optional argument to \subfloat), but instead will
% reference/describe all of them (a), (b), etc., within the main caption.


% An example of a floating table. Note that, for IEEE style tables, the 
% \caption command should come BEFORE the table. Table text will default to
% \footnotesize as IEEE normally uses this smaller font for tables.
% The \label must come after \caption as always.
%
%\begin{table}[!t]
%% increase table row spacing, adjust to taste
%\renewcommand{\arraystretch}{1.3}
% if using array.sty, it might be a good idea to tweak the value of
% \extrarowheight as needed to properly center the text within the cells
%\caption{An Example of a Table}
%\label{table_example}
%\centering
%% Some packages, such as MDW tools, offer better commands for making tables
%% than the plain LaTeX2e tabular which is used here.
%\begin{tabular}{|c||c|}
%\hline
%One & Two\\
%\hline
%Three & Four\\
%\hline
%\end{tabular}
%\end{table}


% Note that IEEE does not put floats in the very first column - or typically
% anywhere on the first page for that matter. Also, in-text middle ("here")
% positioning is not used. Most IEEE journals/conferences use top floats
% exclusively. Note that, LaTeX2e, unlike IEEE journals/conferences, places
% footnotes above bottom floats. This can be corrected via the \fnbelowfloat
% command of the stfloats package.



\section{}

% conference papers do not normally have an appendix


% use section* for acknowledgement
\section{Conclusion}
From the current situation, cloud computing technologies are still immature, which may lead to problems in service management and usability. However  the potential of cloud computing benefit will make it interesting and therefore many people will participate in the progress and development of it. There will be more challenges on this technology, but it will not make it stop. Furthermore it will growth just like the internet today and will be commonly use in the near future.

\section*{Acknowledgment}
The Researcher Mohammed Altemimi would like to thank Dr. Mohd Zamri and all My classmate to listen ...

% trigger a \newpage just before the given reference
% number - used to balance the columns on the last page
% adjust value as needed - may need to be readjusted if
% the document is modified later
%\IEEEtriggeratref{8}
% The "triggered" command can be changed if desired:
%\IEEEtriggercmd{\enlargethispage{-5in}}

% references section

% can use a bibliography generated by BibTeX as a .bbl file
% BibTeX documentation can be easily obtained at:
% http://www.ctan.org/tex-archive/biblio/bibtex/contrib/doc/
% The IEEEtran BibTeX style support page is at:
% http://www.michaelshell.org/tex/ieeetran/bibtex/
%\bibliographystyle{IEEEtran}
% argument is your BibTeX string definitions and bibliography database(s)
%\bibliography{IEEEabrv,../bib/paper}
%
% <OR> manually copy in the resultant .bbl file
% set second argument of \begin to the number of references
% (used to reserve space for the reference number labels box)
\begin{thebibliography}{1}

\bibitem{IEEEhowto:kopka}
Microsoft, \emph{A Guide to Network Infrastrucure \LaTeX}, 3rd~ed.\hskip , \emph, \relax.

\end{thebibliography}



% that's all folks
\end{document}


